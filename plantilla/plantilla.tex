\documentclass[12 pt,a4paper]{article}

\usepackage[utf8]{inputenc}
\usepackage[catalan]{babel}

\usepackage[margin=2cm,top=2cm,includefoot]{geometry}
\usepackage{graphicx}
\usepackage{fancyhdr}
\usepackage{xcolor}
\usepackage[hidelinks]{hyperref}

% Variables
\newcommand{\datasessio}{18.05.2021}
\newcommand{\logo}{images/logo.png}
\newcommand{\participants}{\begin{minipage}[b]{10 cm}
	\large{\textbf{Participants:\\}}
			\begin{itemize}
				\item Equip coordinador:
				\begin{itemize}
					\item Vicenç Palmero (delegat de centre)
					\item Erik Solé (tresorer)
				\end{itemize}
				\item Membres:
				\begin{itemize}
						\item Daniel Díaz 
						\item Marc Armangué 
						\item Albert Sumalla 
						\item Aniol Martí
						\item Laura del Río
				\end{itemize}
			\end{itemize}
\end{minipage}}

% Altres opcions de formats
\addto\captionscatalan{\renewcommand{\contentsname}{Ordre del dia}}

% Configuració de la capçalera i peu de pàgina
\setlength{\headheight}{40 pt}
\pagestyle{fancy}
\fancyhf{}
\renewcommand{\headrulewidth}{2 pt}
\lhead{Delegació d'Alumnes de Telecomunicacions}
\rhead{\includegraphics[width=2 cm,keepaspectratio]{\logo}}
\renewcommand{\footrulewidth}{2 pt}
\lfoot{\textit{UPC Campus Nord, Edifici Omega, Despatx 002\\+34 934 137 670 / dat@dat.upc.edu}}
\rfoot{\textit{\thepage}}

\begin{document}

% Portada
\begin{titlepage}
	\title{\textbf{Acta del Consell de la DAT}}
	\author{}
	\date{\datasessio}
	\maketitle
	\begin{center}
	\vfill
	\includegraphics[width=4cm,keepaspectratio]{\logo}
	\vfill
		\fbox{\participants}
	\end{center}
	\vfill
	\thispagestyle{empty}
\end{titlepage}

% Índex
\tableofcontents
\thispagestyle{empty}
\newpage

\section{Aprovació de l'acta de la sessió anterior}

El Dani presenta l'acta de l'anterior Ple de la DAT, però ràpidament el Vicenç l'indica que l'acta que s'ha de presentar és el de l'anterior Consell. A la propera reunió s'aprovaran les actes pendents.

\section{Informe de l'equip coordinador}

Es necessita algun professor per impartir l’aula lliure de CAVEC. Ara mateix la única opció disponible és l’Aniol, membre present de la DAT. El problema és que no pot cobrar cap beca millot, pel qual la prioritat seria buscar a algú altre. Tot i així, es manté a l’Aniol com a últim recurs. En el pitjor dels casos, es podria posar a una altra persona per cobrar la beca, que l'Aniol doni la classe i donar-li els diners.

\section{Reunió amb la vicerrectora}

Vicenç exposa la reunió que el CdE va tenir amb la vicerectora.\\
El Marc, que estava present a la reunió, parla sobre l’actitud de la vra. A més diu que el possible posicionament de la DAT s’hauria de parlar amb tot el ple i votar-lo. A més, veu molt productives aquestes reunions, ja que creu que poden ser productives, i que aquest possible posicionament s’hauria de produir després de les reunions.\\
Albert pregunta si és normal que els candidats parlin amb les associacions. Es mostra contrari a la reunió.\\
El Vicens exposa que l’actual rector va guanyar gràcies al vot de l’estudiantat, cosa que no va agradar al claustre.\\
L’Albert pregunta si seria possible un sondeig per saber l’opinió general, tot i que la resta de membres no ho veuen molt factible.

\section{Memòria DAT Q1}

Divendres s’ha de presentar la memòria de la DAT per demanar les beques internes de cara al proper quatrimestre de tardor del curs 2021-2022. Acte seguit, el Vicenç mostra el pressupost disponible per aquestes. Les beques que derivin d’aquest pressupost es decidiran al proper ple.\\
L’Aniol destaca que algú s’ha d’encarregar del servidor, ja que ell ha de deixar aquesta tasca.\\
L’Albert li recorda a l'Aniol que ha de redactar un HOWTO, ja que la persona que s'aculli a aquesta beca no tindrà un relleu fet personalment per ell.

\section{Proposta per la pàgina web}

L'Albert ens mostra la seva llista de coses per fer al WORDPRESS:
\begin{itemize}
	\item Afegir traduccions a català, castellà i anglè.s
	\item Posar baferades a la secció de membres i  una explicació de les beques que tenim.
	\item Animació pujada i baixada a les taules on es mostren els apunts per saber que ja s'ha canviat de grau i assignatura.
	\item Alinear exàmens finals amb parcials i apunts.
	\item Afegir números als exàmens i apunts per a que estiguin ordenats. 
	\item Afegir secció d'erasmus a FAQS.
	\item Mostrar els membres a l'apartat de membres. 
	\item Al mateix apartat, que apargeui també un apartat de col·labora.
	\item A projectes afegir beques, i explicar cada una.
\end{itemize}

\section{Proposta d'activitat per l'últim dia de classes}

El Vicenç pregunta qui estaria disposat a ajudar en cas que es pugui realitzar alguna activitat, com ara una sindriada o una cervesada. Es compta amb el suport del Vicenç, Erik, Dani, Arnau, Albert.\\
Finalment, es conclou que no és factible colaborar amb altres delegacions en cap cas, ja que una major difusió comportarà més aglomeracions.

\section{Comentaris sobre el darrer Ple del CdE}

Es torna a fer referència a la reunió amb la vicerrectora, esmentada a la secció 3. Sobre tot, es destaca el tema dels treballs finals de grau i de màster (TFG i TFM). El Marc va proposar el debat sobre la possible no presencialitat a les presentacions, 
La resolució hauria de ser 100\% online o presencial. 
L’Aniol podria presentar la proposta a la CP i s’hauria de redactar la proposta de cara al proper mes.

\section{Easter egg a la pàgina web}

Es proposa fer un concurs per trobar l'easter egg esmentat al punt 5 i regalar merxandatge a qui ho trobi abans.

\section{Reunió amb el director}

El proper dimecres, 5 de maig, el Vicenç tinrdà una reunió amb el director. Es parlarà sobre l’aula lliure, el projecte de la sindriada (punt 6) i el possible posicionament a la campanya (punt 3). No s’ha de parlar dels TFG (punt 7)

\section{Torn obert de paraula}

\end{document}
